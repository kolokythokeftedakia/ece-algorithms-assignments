Στην παρούσα αναφορά έχουν υιοθετηθεί οι παρακάτω συμβολισμοί.\par

\newcolumntype{C}{>{\centering\arraybackslash}m{4.0cm}}
\newcolumntype{D}{>{\centering\arraybackslash}m{5.0cm}}
\renewcommand{\arraystretch}{1.5}
\newcommand{\mono}[1]{$\mathtt{#1}$}
\begin{table}[H]
	\begin{center}
		\begin{tabular}{|D|D|}
			\hline
			\textbf{Συμβολισμός ψευδογλώσσας}& \textbf{Ερμηνεία}\\\hline\hline
			\mono{a\leftarrow b}&  καταχώρηση του \mono{b} στο \mono{a}\\\hline
			\mono{++a} & προσαύξηση του \mono{a} κατά \mono{1}\\\hline
			\mono{\mathbf{\texttt{uint}}\;a} & μη αρνητικός ακέραιος αριθμός\\\hline
			\mono{\mathbf{\texttt{array}}\;a} & μία μονοδοδιάστατη, zero-indexed, διατεταγμένη λίστα στοιχείων, δηλαδή ένα διάνυσμα στοιχείων\\\hline
			\mono{arr[a:b]} & διαμέριση του πίνακα \mono{arr} στην οποία ανήκουν τα στοιχεία \mono{\left\{arr[a],\ldots,arr[b-1]\right\}} με τη σειρά που εμφανίζονται στον \mono{arr}\\\hline
                \mono{\texttt{code}\quad//comment} & ό,τι βρίσκεται δεξιά των \mono{//} είναι σχόλιο (in-line comment)\\\hline
		\end{tabular}
	\end{center}
	\vspace*{-10pt}
	\caption{Συμβολισμοί σε περιβάλλον ψευδογλώσσας.}
	\vspace*{+10pt}
\end{table}

\begin{table}[H]
	\begin{center}
		\begin{tabular}{|C|C|}
			\hline
			\textbf{Συμβολισμός συναρτήσεων}&\textbf{Ερμηνεία}\\\hline\hline
			$\lg{a}$ & $\log_{2}{a}$\\\hline
			$\ln{a}$ & $\log_{e}{a}$\\\hline
			$\log{a}$ & $\log_{10}{a}$\\\hline
			$\left\lceil x\right\rceil$ & συνάρτηση ceiling: $\min\{n|n\geqslant x,\;\text{integer}\;n\}$ \cit{ConcreteMath}\\\hline
			$\left\lfloor x\right\rfloor$ & συνάρτηση floor: $\max\{n|n\leqslant x,\;\text{integer}\;n\}$ \cit{ConcreteMath}\\\hline
		\end{tabular}
	\end{center}
	\vspace*{-10pt}
	\caption{Συμβολισμοί συνήθων συναρτήσεων.}
	\vspace*{+10pt}
\end{table}
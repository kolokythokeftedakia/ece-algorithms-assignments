\newcommand{\emptyCell}{\textcolor{white}{X}}
\begin{tikzpicture}[font=\ttfamily,
	array/.style={matrix of nodes,nodes={draw, minimum size=6mm, fill=white},column sep=-\pgflinewidth, row sep=0.75mm, nodes in empty cells,
	row 1/.style={nodes={draw=none, fill=none, minimum size=4mm}}}]

	\matrix[array] (array) {
		1 & 2 & 3 & 4 & 5 & 6\\
		1 & 0 & 3 & 1 & 3 & 1\\};

	\begin{scope}[on background layer]
	\fill[blue!20] (array-1-1.north west) rectangle (array-1-6.south east);
	\end{scope}

	\draw (array-1-1.west) node [left]{indices};
	\draw (array-2-1.west) node [left]{$T$};

\end{tikzpicture}

\begin{enumerate}[label=\textbf{Βήμα \arabic*.}]
	\item Δημιουργία και αρχικοποίηση του πίνακα $H$ σε μηδενικά --- γραμμές $1$ έως και $3$:\\\begin{center}
		\begin{tikzpicture}[font=\ttfamily,
			array/.style={matrix of nodes,nodes={draw, minimum size=6mm, fill=white},column sep=-\pgflinewidth, row sep=0.75mm, nodes in empty cells,
			row 1/.style={nodes={draw=none, fill=none, minimum size=4mm}}}]

			\matrix[array] (array) {
				0 & 1 & 2 & 3\\
				0 & 0 & 0 & 0\\};

			\begin{scope}[on background layer]
			\fill[red!20] (array-1-1.north west) rectangle (array-1-4.south east);
			\end{scope}

			\draw (array-1-1.west) node [left]{indices};
			\draw (array-2-1.west) node [left]{$H$};

		\end{tikzpicture}\end{center}
	\item Σαρώνουμε τον πίνακα $T$ και για κάθε ακέραιο $j$ που συναντάμε αυξάνουμε κατά ένα $1$ το περιεχόμενο της θέσης του $H$ με index $T[j]$. Τελικά, σε κάθε θέση $j$ του πίνακα $H$ θα υπάρχει αποθηκευμένη η συχνότητα εμφάνισης του $j$ στον πίνακα $T$. Δηλαδή, για κάθε ακέραιο $j$ κάνουμε $H[T[j]]=H[T[j]]+1$, $j\in[1,n]$.
		\begin{center}
			\begin{tikzpicture}[font=\ttfamily,
				array/.style={matrix of nodes,nodes={draw, minimum size=6mm, fill=white},column sep=-\pgflinewidth, row sep=0.75mm, nodes in empty cells,
				row 1/.style={nodes={draw=none, fill=none, minimum size=4mm}},
				row 1 column 2/.style={nodes={draw}}}]
				\matrix[array] (array) {
				0 & 1 & 2 & 3\\
				0 & 1 & 0 & 0\\};
				\begin{scope}[on background layer]
					\fill[red!20] (array-1-1.north west) rectangle (array-1-4.south east);
				\end{scope}
				\draw (array-1-2.north)--++(90:3mm) node [above] (second) {$T[1]$ for $j=1$};
				\draw (array-1-1.west) node [left]{indices};
				\draw (array-2-1.west) node [left]{$H$};
			\end{tikzpicture}\hspace*{24pt}
			\begin{tikzpicture}[font=\ttfamily,
				array/.style={matrix of nodes,nodes={draw, minimum size=6mm, fill=white},column sep=-\pgflinewidth, row sep=0.75mm, nodes in empty cells,
				row 1/.style={nodes={draw=none, fill=none, minimum size=4mm}},
				row 1 column 1/.style={nodes={draw}}}]
				\matrix[array] (array) {
				0 & 1 & 2 & 3\\
				1 & 1 & 0 & 0\\};
				\begin{scope}[on background layer]
					\fill[red!20] (array-1-1.north west) rectangle (array-1-4.south east);
				\end{scope}
				\draw (array-1-1.north)--++(90:3mm) node [above] (first) {$T[2]$ for $j=2$};
				\draw (array-1-1.west) node [left]{indices};
				\draw (array-2-1.west) node [left]{$H$};
			\end{tikzpicture}\hspace*{24pt}
			\begin{tikzpicture}[font=\ttfamily,
				array/.style={matrix of nodes,nodes={draw, minimum size=6mm, fill=white},column sep=-\pgflinewidth, row sep=0.75mm, nodes in empty cells,
				row 1/.style={nodes={draw=none, fill=none, minimum size=4mm}},
				row 1 column 4/.style={nodes={draw}}}]
				\matrix[array] (array) {
				0 & 1 & 2 & 3\\
				1 & 1 & 0 & 1\\};
				\begin{scope}[on background layer]
					\fill[red!20] (array-1-1.north west) rectangle (array-1-4.south east);
				\end{scope}
				\draw (array-1-4.north)--++(90:3mm) node [above] (fourth) {$T[3]$ for $j=3$};
				\draw (array-1-1.west) node [left]{indices};
				\draw (array-2-1.west) node [left]{$H$};
			\end{tikzpicture}\\\vspace*{24pt}
			\begin{tikzpicture}[font=\ttfamily,
				array/.style={matrix of nodes,nodes={draw, minimum size=6mm, fill=white},column sep=-\pgflinewidth, row sep=0.75mm, nodes in empty cells,
				row 1/.style={nodes={draw=none, fill=none, minimum size=4mm}},
				row 1 column 2/.style={nodes={draw}}}]
				\matrix[array] (array) {
				0 & 1 & 2 & 3\\
				1 & 2 & 0 & 1\\};
				\begin{scope}[on background layer]
					\fill[red!20] (array-1-1.north west) rectangle (array-1-4.south east);
				\end{scope}
				\draw (array-1-2.north)--++(90:3mm) node [above] (second) {$T[4]$ for $j=4$};
				\draw (array-1-1.west) node [left]{indices};
				\draw (array-2-1.west) node [left]{$H$};
			\end{tikzpicture}\hspace*{18pt}
			\begin{tikzpicture}[font=\ttfamily,
				array/.style={matrix of nodes,nodes={draw, minimum size=6mm, fill=white},column sep=-\pgflinewidth, row sep=0.75mm, nodes in empty cells,
				row 1/.style={nodes={draw=none, fill=none, minimum size=4mm}},
				row 1 column 4/.style={nodes={draw}}}]
				\matrix[array] (array) {
				0 & 1 & 2 & 3\\
				1 & 2 & 0 & 2\\};
				\begin{scope}[on background layer]
					\fill[red!20] (array-1-1.north west) rectangle (array-1-4.south east);
				\end{scope}
				\draw (array-1-4.north)--++(90:3mm) node [above] (T[j]) {$T[5]$ for $j=5$};
				\draw (array-1-1.west) node [left]{indices};
				\draw (array-2-1.west) node [left]{$H$};
			\end{tikzpicture}
			\begin{tikzpicture}[font=\ttfamily,
				array/.style={matrix of nodes,nodes={draw, minimum size=6mm, fill=white},column sep=-\pgflinewidth, row sep=0.75mm, nodes in empty cells,
				row 1/.style={nodes={draw=none, fill=none, minimum size=4mm}},
				row 1 column 2/.style={nodes={draw}}}]
				\matrix[array] (array) {
				0 & 1 & 2 & 3\\
				1 & 3 & 0 & 2\\};
				\begin{scope}[on background layer]
					\fill[red!20] (array-1-1.north west) rectangle (array-1-4.south east);
				\end{scope}
				\draw (array-1-2.north)--++(90:3mm) node [above] (T[j]) {$T[6]$ for $j=6$};
				\draw (array-1-1.west) node [left]{indices};
				\draw (array-2-1.west) node [left]{$H$};
			\end{tikzpicture}
		\end{center}
	\item Στον πίνακα $H$ του προηγούμενου βήματος, ξεκινώντας από την δεύτερη θέση, δηλαδή το $H[1]$, προσθέτουμε στο ήδη υπάρχον περιεχόμενο της εκάστοτε θέσης του πίνακα $H$, το περιεχόμενο της προηγούμενης θέσης (αυτός είναι και ο λόγος για τον οποίο αφήνουμε αμετάβλητο το $H[0]$) --- γραμμές $7$ έως και $9$:
		\begin{center}
			\begin{tikzpicture}[font=\ttfamily,
				array/.style={matrix of nodes,nodes={draw, minimum size=6mm, fill=white},column sep=-\pgflinewidth, row sep=0.75mm, nodes in empty cells,
				row 1/.style={nodes={draw=none, fill=none, minimum size=4mm}},
				row 1 column 2/.style={nodes={draw}}}]
				\matrix[array] (array) {
				0 & 1 & 2 & 3\\
				1 & 3+1 & 0 & 2\\};
				\begin{scope}[on background layer]
					\fill[red!20] (array-1-1.north west) rectangle (array-1-4.south east);
				\end{scope}
				\draw (array-1-2.north)--++(90:3mm) node [above] (second) {$i=1$};
				\draw (array-1-1.west) node [left]{indices};
				\draw (array-2-1.west) node [left]{$H$};
			\end{tikzpicture}\hspace*{24pt}
			\begin{tikzpicture}[font=\ttfamily,
				array/.style={matrix of nodes,nodes={draw, minimum size=6mm, fill=white},column sep=-\pgflinewidth, row sep=0.75mm, nodes in empty cells,
				row 1/.style={nodes={draw=none, fill=none, minimum size=4mm}},
				row 1 column 2/.style={nodes={draw}}}]
				\matrix[array] (array) {
				0 & 1 & 2 & 3\\
				1 & 4 & 0+4 & 2\\};
				\begin{scope}[on background layer]
					\fill[red!20] (array-1-1.north west) rectangle (array-1-4.south east);
				\end{scope}
				\draw (array-1-2.north)--++(90:3mm) node [above] (first) {$i=2$};
				\draw (array-1-1.west) node [left]{indices};
				\draw (array-2-1.west) node [left]{$H$};
			\end{tikzpicture}\hspace*{24pt}
			\begin{tikzpicture}[font=\ttfamily,
				array/.style={matrix of nodes,nodes={draw, minimum size=6mm, fill=white},column sep=-\pgflinewidth, row sep=0.75mm, nodes in empty cells,
				row 1/.style={nodes={draw=none, fill=none, minimum size=4mm}},
				row 1 column 4/.style={nodes={draw}}}]
				\matrix[array] (array) {
				0 & 1 & 2 & 3\\
				1 & 4 & 4 & 2+4\\};
				\begin{scope}[on background layer]
					\fill[red!20] (array-1-1.north west) rectangle (array-1-4.south east);
				\end{scope}
				\draw (array-1-4.north)--++(90:3mm) node [above] (fourth) {$i=3$};
				\draw (array-1-1.west) node [left]{indices};
				\draw (array-2-1.west) node [left]{$H$};
			\end{tikzpicture}
		\end{center}
		Τελικά, θα είναι
		\begin{center}
			\begin{tikzpicture}[font=\ttfamily,
				array/.style={matrix of nodes,nodes={draw, minimum size=6mm, fill=white},column sep=-\pgflinewidth, row sep=0.75mm, nodes in empty cells,
				row 1/.style={nodes={draw=none, fill=none, minimum size=4mm}}}]
				\matrix[array] (array) {
				0 & 1 & 2 & 3\\
				1 & 4 & 4 & 6\\};
				\begin{scope}[on background layer]
					\fill[red!20] (array-1-1.north west) rectangle (array-1-4.south east);
				\end{scope}
				\draw (array-1-1.west) node [left]{indices};
				\draw (array-2-1.west) node [left]{$H$};
			\end{tikzpicture}
		\end{center}
		Η νέα μορφή του πίνακα $H$ μας δίνει την πληροφορία \textsl{σε ποια θέση του ταξινομημένου πίνακα θα τοποθετήσουμε για τελευταία φορά\footnote{Με την έκφραση \textsl{«τελευταία φορά»} εννοούμε πως στην περίπτωση όπου κάποιος ακέραιος, $x$, εμφανίζεται στον $T$ περισσότερες από μία φορές, τότε το στοιχείο με τιμή $x$ το οποίο έχει μεγαλύτερο index στον $T$ από όλα τα υπόλοιπα στοιχεία του $T$ με τιμή $x$, θα εμφανιστεί στον $S$ στην θέση--index που του αντιστοιχεί σύμφωνα με τον $H$ μετά το πέρας του \textbf{Βήματος 3}.} κάθε ακέραιο (index του πίνακα $H$).}\newpage
	\item Σε έναν νέο πίνακα $S$, μεγέθους $n$, αποθηκεύουμε το ταξινομημένο αποτέλεσμα. Ο μετρητής του βρόχου επανάληψης των γραμμών $10$ έως και $13$ ξεκινάει από το $n$ και μειώνεται έως και την τιμή $1$. Χρησιμοποιούμε ως δείκτη για την πρόσβαση στον πίνακα $H$ την τιμή που είναι αποθηκευμένη στο $T[j]$, δηλαδή διαβάζουμε την τιμή $H[T[j]]$ η οποία μας δίνει την θέση της τελευταίας εμφάνισης του ακεραίου $T[j]$ στον ταξινομημένο πίνακα $S$ (σε κάθε iteration). Επομένως, θα είναι $S[H[T[j]]]=T[j]$.

		\begin{itemize}
			\item $j=6$:
				\begin{center}
					\begin{tikzpicture}[font=\ttfamily,
						array/.style={matrix of nodes,nodes={draw, minimum size=6mm, fill=white},column sep=-\pgflinewidth, row sep=0.75mm, nodes in empty cells,
						row 1/.style={nodes={draw=none, fill=none, minimum size=4mm}},
						row 1 column 6/.style={nodes={draw}}}]
						\matrix[array] (array) {
						1 & 2 & 3 & 4 & 5 & 6\\
						1 & 0 & 3 & 1 & 3 & 1\\};
						\begin{scope}[on background layer]
							\fill[blue!20] (array-1-1.north west) rectangle (array-1-6.south east);
						\end{scope}
						\draw (array-1-6.north)--++(90:3mm) node [above] (fourth) {$j=6$};
						\draw (array-1-1.west) node [left]{indices};
						\draw (array-2-1.west) node [left]{$T$};
					\end{tikzpicture}\hspace*{24pt}
					\begin{tikzpicture}[font=\ttfamily,
						array/.style={matrix of nodes,nodes={draw, minimum size=6mm, fill=white},column sep=-\pgflinewidth, row sep=0.75mm, nodes in empty cells,
						row 1/.style={nodes={draw=none, fill=none, minimum size=4mm}},
						row 1 column 2/.style={nodes={draw}}}]
						\matrix[array] (array) {
						0 & 1 & 2 & 3\\
						1 & 4 & 4 & 6\\};
						\begin{scope}[on background layer]
							\fill[red!20] (array-1-1.north west) rectangle (array-1-4.south east);
						\end{scope}
						\draw (array-1-2.north)--++(90:3mm) node [above] (fourth) {$T[6]$};
						\draw (array-1-1.west) node [left]{indices};
						\draw (array-2-1.west) node [left]{$H$};
					\end{tikzpicture}\\\vspace*{18pt}\begin{tikzpicture}[font=\ttfamily,
						array/.style={matrix of nodes,nodes={draw, minimum size=6mm, fill=white},column sep=-\pgflinewidth, row sep=0.75mm, nodes in empty cells,
						row 1/.style={nodes={draw=none, fill=none, minimum size=4mm}},
						nodes in empty cells,
						row 2/.style={nodes={draw, fill=white, minimum size=6mm}},
						row 1 column 4/.style={nodes={draw}}}]
						\matrix[array] (array) {
						1 & 2 & 3 & 4 & 5 & 6\\
						\emptyCell&\emptyCell&\emptyCell&1&\emptyCell&\emptyCell\\};
						\begin{scope}[on background layer]
							\fill[green!20] (array-1-1.north west) rectangle (array-1-6.south east);
						\end{scope}
						\draw (array-1-4.north)--++(90:3mm) node [above] (fourth) {$j=6:H[T[6]]$};
						\draw (array-1-1.west) node [left]{indices};
						\draw (array-2-1.west) node [left]{$S$};
					\end{tikzpicture}
				\end{center}
				Αφού τοποθετήσουμε τον πρώτο ακέραιο στον ταξινομημένο πίνακα $S$ πρέπει να μειώσουμε το περιεχόμενο στην αντίστοιχη καταχώρηση του $H$ κατά $1$, διότι την επόμενη φορά που θα συναντήσουμε τον συγκεκριμένο ακέραιο στον πίνακα $T$ θα πρέπει να τοποθετηθεί στην αμέσως προηγούμενη θέση στον $S$. Δηλαδή, η νέα μορφή του $H$ είναι
				\begin{center}
					\begin{tikzpicture}[font=\ttfamily,
						array/.style={matrix of nodes,nodes={draw, minimum size=6mm, fill=none},column sep=-\pgflinewidth, row sep=0.75mm, nodes in empty cells,
						row 1/.style={nodes={draw=none, fill=none, minimum size=4mm}}}]
						\matrix[array] (array) {
						0 & 1 & 2 & 3\\
						1 & 4-1 & 4 & 6\\};
						\begin{scope}[on background layer]
							\fill[red!20] (array-1-1.north west) rectangle (array-1-4.south east);
						\end{scope}
						\draw (array-1-1.west) node [left]{indices};
						\draw (array-2-1.west) node [left]{$H$};
					\end{tikzpicture}
				\end{center}
			\item $j=5$:
				\begin{center}
					\begin{tikzpicture}[font=\ttfamily,
						array/.style={matrix of nodes,nodes={draw, minimum size=6mm, fill=white},column sep=-\pgflinewidth, row sep=0.75mm, nodes in empty cells,
						row 1/.style={nodes={draw=none, fill=none, minimum size=4mm}},
						row 1 column 5/.style={nodes={draw}}}]
						\matrix[array] (array) {
						1 & 2 & 3 & 4 & 5 & 6\\
						1 & 0 & 3 & 1 & 3 & 1\\};
						\begin{scope}[on background layer]
							\fill[blue!20] (array-1-1.north west) rectangle (array-1-6.south east);
						\end{scope}
						\draw (array-1-5.north)--++(90:3mm) node [above] (fourth) {$j=5$};
						\draw (array-1-1.west) node [left]{indices};
						\draw (array-2-1.west) node [left]{$T$};
					\end{tikzpicture}\hspace*{24pt}
					\begin{tikzpicture}[font=\ttfamily,
						array/.style={matrix of nodes,nodes={draw, minimum size=6mm, fill=white},column sep=-\pgflinewidth, row sep=0.75mm, nodes in empty cells,
						row 1/.style={nodes={draw=none, fill=none, minimum size=4mm}},
						row 1 column 4/.style={nodes={draw}}}]
						\matrix[array] (array) {
						0 & 1 & 2 & 3\\
						1 & 3 & 4 & 6\\};
						\begin{scope}[on background layer]
							\fill[red!20] (array-1-1.north west) rectangle (array-1-4.south east);
						\end{scope}
						\draw (array-1-4.north)--++(90:3mm) node [above] (fourth) {$T[5]$};
						\draw (array-1-1.west) node [left]{indices};
						\draw (array-2-1.west) node [left]{$H$};
					\end{tikzpicture}\\\vspace*{18pt}\begin{tikzpicture}[font=\ttfamily,
						array/.style={matrix of nodes,nodes={draw, minimum size=6mm, fill=white},column sep=-\pgflinewidth, row sep=0.75mm, nodes in empty cells,
						row 1/.style={nodes={draw=none, fill=none, minimum size=4mm}},
						nodes in empty cells,
						row 2/.style={nodes={draw, fill=white, minimum size=6mm}},
						row 1 column 6/.style={nodes={draw}}}]
						\matrix[array] (array) {
						1 & 2 & 3 & 4 & 5 & 6\\
						\emptyCell&\emptyCell&\emptyCell&1&\emptyCell&3\\};
						\begin{scope}[on background layer]
							\fill[green!20] (array-1-1.north west) rectangle (array-1-6.south east);
						\end{scope}
						\draw (array-1-6.north)--++(90:3mm) node [above] (fourth) {$j=5:H[T[5]]$};
						\draw (array-1-1.west) node [left]{indices};
						\draw (array-2-1.west) node [left]{$S$};
					\end{tikzpicture}
				\end{center}

			\item $j=4$:
			\begin{center}
				\begin{tikzpicture}[font=\ttfamily,
					array/.style={matrix of nodes,nodes={draw, minimum size=6mm, fill=white},column sep=-\pgflinewidth, row sep=0.75mm, nodes in empty cells,
					row 1/.style={nodes={draw=none, fill=none, minimum size=4mm}},
					row 1 column 4/.style={nodes={draw}}}]
					\matrix[array] (array) {
					1 & 2 & 3 & 4 & 5 & 6\\
					1 & 0 & 3 & 1 & 3 & 1\\};
					\begin{scope}[on background layer]
						\fill[blue!20] (array-1-1.north west) rectangle (array-1-6.south east);
					\end{scope}
					\draw (array-1-4.north)--++(90:3mm) node [above] (fourth) {$j=4$};
					\draw (array-1-1.west) node [left]{indices};
					\draw (array-2-1.west) node [left]{$T$};
				\end{tikzpicture}\hspace*{24pt}
				\begin{tikzpicture}[font=\ttfamily,
					array/.style={matrix of nodes,nodes={draw, minimum size=6mm, fill=white},column sep=-\pgflinewidth, row sep=0.75mm, nodes in empty cells,
					row 1/.style={nodes={draw=none, fill=none, minimum size=4mm}},
					row 1 column 2/.style={nodes={draw}}}]
					\matrix[array] (array) {
					0 & 1 & 2 & 3\\
					1 & 3 & 4 & 5\\};
					\begin{scope}[on background layer]
						\fill[red!20] (array-1-1.north west) rectangle (array-1-4.south east);
					\end{scope}
					\draw (array-1-2.north)--++(90:3mm) node [above] (fourth) {$T[4]$};
					\draw (array-1-1.west) node [left]{indices};
					\draw (array-2-1.west) node [left]{$H$};
				\end{tikzpicture}\\\vspace*{18pt}\begin{tikzpicture}[font=\ttfamily,
					array/.style={matrix of nodes,nodes={draw, minimum size=6mm, fill=white},column sep=-\pgflinewidth, row sep=0.75mm, nodes in empty cells,
					row 1/.style={nodes={draw=none, fill=none, minimum size=4mm}},
					nodes in empty cells,
					row 2/.style={nodes={draw, fill=white, minimum size=6mm}},
					row 1 column 3/.style={nodes={draw}}}]
					\matrix[array] (array) {
					1 & 2 & 3 & 4 & 5 & 6\\
					\emptyCell&\emptyCell&1&1&\emptyCell&3\\};
					\begin{scope}[on background layer]
						\fill[green!20] (array-1-1.north west) rectangle (array-1-6.south east);
					\end{scope}
					\draw (array-1-3.north)--++(90:3mm) node [above] (fourth) {$j=4:H[T[4]]$};
					\draw (array-1-1.west) node [left]{indices};
					\draw (array-2-1.west) node [left]{$S$};
				\end{tikzpicture}
			\end{center}\newpage
			\item $j=3$:
			\begin{center}
				\begin{tikzpicture}[font=\ttfamily,
					array/.style={matrix of nodes,nodes={draw, minimum size=6mm, fill=white},column sep=-\pgflinewidth, row sep=0.75mm, nodes in empty cells,
					row 1/.style={nodes={draw=none, fill=none, minimum size=4mm}},
					row 1 column 3/.style={nodes={draw}}}]
					\matrix[array] (array) {
					1 & 2 & 3 & 4 & 5 & 6\\
					1 & 0 & 3 & 1 & 3 & 1\\};
					\begin{scope}[on background layer]
						\fill[blue!20] (array-1-1.north west) rectangle (array-1-6.south east);
					\end{scope}
					\draw (array-1-3.north)--++(90:3mm) node [above] (fourth) {$j=3$};
					\draw (array-1-1.west) node [left]{indices};
					\draw (array-2-1.west) node [left]{$T$};
				\end{tikzpicture}\hspace*{24pt}
				\begin{tikzpicture}[font=\ttfamily,
					array/.style={matrix of nodes,nodes={draw, minimum size=6mm, fill=white},column sep=-\pgflinewidth, row sep=0.75mm, nodes in empty cells,
					row 1/.style={nodes={draw=none, fill=none, minimum size=4mm}},
					row 1 column 4/.style={nodes={draw}}}]
					\matrix[array] (array) {
					0 & 1 & 2 & 3\\
					1 & 2 & 4 & 5\\};
					\begin{scope}[on background layer]
						\fill[red!20] (array-1-1.north west) rectangle (array-1-4.south east);
					\end{scope}
					\draw (array-1-4.north)--++(90:3mm) node [above] (fourth) {$T[3]$};
					\draw (array-1-1.west) node [left]{indices};
					\draw (array-2-1.west) node [left]{$H$};
				\end{tikzpicture}\\\vspace*{18pt}\begin{tikzpicture}[font=\ttfamily,
					array/.style={matrix of nodes,nodes={draw, minimum size=6mm, fill=white},column sep=-\pgflinewidth, row sep=0.75mm, nodes in empty cells,
					row 1/.style={nodes={draw=none, fill=none, minimum size=4mm}},
					nodes in empty cells,
					row 2/.style={nodes={draw, fill=white, minimum size=6mm}},
					row 1 column 5/.style={nodes={draw}}}]
					\matrix[array] (array) {
					1 & 2 & 3 & 4 & 5 & 6\\
					\emptyCell&\emptyCell&1&1&3&3\\};
					\begin{scope}[on background layer]
						\fill[green!20] (array-1-1.north west) rectangle (array-1-6.south east);
					\end{scope}
					\draw (array-1-5.north)--++(90:3mm) node [above] (fourth) {$j=3:H[T[3]]$};
					\draw (array-1-1.west) node [left]{indices};
					\draw (array-2-1.west) node [left]{$S$};
				\end{tikzpicture}
			\end{center}
			\item $j=2$:
			\begin{center}
				\begin{tikzpicture}[font=\ttfamily,
					array/.style={matrix of nodes,nodes={draw, minimum size=6mm, fill=white},column sep=-\pgflinewidth, row sep=0.75mm, nodes in empty cells,
					row 1/.style={nodes={draw=none, fill=none, minimum size=4mm}},
					row 1 column 2/.style={nodes={draw}}}]
					\matrix[array] (array) {
					1 & 2 & 3 & 4 & 5 & 6\\
					1 & 0 & 3 & 1 & 3 & 1\\};
					\begin{scope}[on background layer]
						\fill[blue!20] (array-1-1.north west) rectangle (array-1-6.south east);
					\end{scope}
					\draw (array-1-2.north)--++(90:3mm) node [above] (fourth) {$j=2$};
					\draw (array-1-1.west) node [left]{indices};
					\draw (array-2-1.west) node [left]{$T$};
				\end{tikzpicture}\hspace*{24pt}
				\begin{tikzpicture}[font=\ttfamily,
					array/.style={matrix of nodes,nodes={draw, minimum size=6mm, fill=white},column sep=-\pgflinewidth, row sep=0.75mm, nodes in empty cells,
					row 1/.style={nodes={draw=none, fill=none, minimum size=4mm}},
					row 1 column 1/.style={nodes={draw}}}]
					\matrix[array] (array) {
					0 & 1 & 2 & 3\\
					1 & 2 & 4 & 4\\};
					\begin{scope}[on background layer]
						\fill[red!20] (array-1-1.north west) rectangle (array-1-4.south east);
					\end{scope}
					\draw (array-1-1.north)--++(90:3mm) node [above] (fourth) {$T[2]$};
					\draw (array-1-1.west) node [left]{indices};
					\draw (array-2-1.west) node [left]{$H$};
				\end{tikzpicture}\\\vspace*{18pt}\begin{tikzpicture}[font=\ttfamily,
					array/.style={matrix of nodes,nodes={draw, minimum size=6mm, fill=white},column sep=-\pgflinewidth, row sep=0.75mm, nodes in empty cells,
					row 1/.style={nodes={draw=none, fill=none, minimum size=4mm}},
					nodes in empty cells,
					row 2/.style={nodes={draw, fill=white, minimum size=6mm}},
					row 1 column 1/.style={nodes={draw}}}]
					\matrix[array] (array) {
					1 & 2 & 3 & 4 & 5 & 6\\
					0&\emptyCell&1&1&3&3\\};
					\begin{scope}[on background layer]
						\fill[green!20] (array-1-1.north west) rectangle (array-1-6.south east);
					\end{scope}
					\draw (array-1-1.north)--++(90:3mm) node [above] (fourth) {$j=2:H[T[2]]$};
					\draw (array-1-1.west) node [left]{indices};
					\draw (array-2-1.west) node [left]{$S$};
				\end{tikzpicture}
			\end{center}
			\item $j=1$:
			\begin{center}
				\begin{tikzpicture}[font=\ttfamily,
					array/.style={matrix of nodes,nodes={draw, minimum size=6mm, fill=white},column sep=-\pgflinewidth, row sep=0.75mm, nodes in empty cells,
					row 1/.style={nodes={draw=none, fill=none, minimum size=4mm}},
					row 1 column 1/.style={nodes={draw}}}]
					\matrix[array] (array) {
					1 & 2 & 3 & 4 & 5 & 6\\
					1 & 0 & 3 & 1 & 3 & 1\\};
					\begin{scope}[on background layer]
						\fill[blue!20] (array-1-1.north west) rectangle (array-1-6.south east);
					\end{scope}
					\draw (array-1-1.north)--++(90:3mm) node [above] (fourth) {$j=1$};
					\draw (array-1-1.west) node [left]{indices};
					\draw (array-2-1.west) node [left]{$T$};
				\end{tikzpicture}\hspace*{24pt}
				\begin{tikzpicture}[font=\ttfamily,
					array/.style={matrix of nodes,nodes={draw, minimum size=6mm, fill=white},column sep=-\pgflinewidth, row sep=0.75mm, nodes in empty cells,
					row 1/.style={nodes={draw=none, fill=none, minimum size=4mm}},
					row 1 column 2/.style={nodes={draw}}}]
					\matrix[array] (array) {
					0 & 1 & 2 & 3\\
					0 & 2 & 4 & 4\\};
					\begin{scope}[on background layer]
						\fill[red!20] (array-1-1.north west) rectangle (array-1-4.south east);
					\end{scope}
					\draw (array-1-2.north)--++(90:3mm) node [above] (fourth) {$T[1]$};
					\draw (array-1-1.west) node [left]{indices};
					\draw (array-2-1.west) node [left]{$H$};
				\end{tikzpicture}\\\vspace*{18pt}\begin{tikzpicture}[font=\ttfamily,
					array/.style={matrix of nodes,nodes={draw, minimum size=6mm, fill=white},column sep=-\pgflinewidth, row sep=0.75mm, nodes in empty cells,
					row 1/.style={nodes={draw=none, fill=none, minimum size=4mm}},
					nodes in empty cells,
					row 2/.style={nodes={draw, fill=white, minimum size=6mm}},
					row 1 column 2/.style={nodes={draw}}}]
					\matrix[array] (array) {
					1 & 2 & 3 & 4 & 5 & 6\\
					0&1&1&1&3&3\\};
					\begin{scope}[on background layer]
						\fill[green!20] (array-1-1.north west) rectangle (array-1-6.south east);
					\end{scope}
					\draw (array-1-2.north)--++(90:3mm) node [above] (fourth) {$j=1:H[T[1]]$};
					\draw (array-1-1.west) node [left]{indices};
					\draw (array-2-1.west) node [left]{$S$};
				\end{tikzpicture}
			\end{center}
		\end{itemize}
		Τελικά, ο πίνακας $H$ είναι
		\begin{center}
			\begin{tikzpicture}[font=\ttfamily,
				array/.style={matrix of nodes,nodes={draw, minimum size=6mm, fill=white},column sep=-\pgflinewidth, row sep=0.75mm, nodes in empty cells,
				row 1/.style={nodes={draw=none, fill=none, minimum size=4mm}}}]
				\matrix[array] (array) {
				0 & 1 & 2 & 3\\
				0 & 1 & 4 & 4\\};
				\begin{scope}[on background layer]
					\fill[red!20] (array-1-1.north west) rectangle (array-1-4.south east);
				\end{scope}
				\draw (array-1-1.west) node [left]{indices};
				\draw (array-2-1.west) node [left]{$H$};
			\end{tikzpicture}
		\end{center}
		και ο ταξινομημένος πίνακας $S$ είναι
		\begin{center}
			\begin{tikzpicture}[font=\ttfamily,
				array/.style={matrix of nodes,nodes={draw, minimum size=6mm, fill=white},column sep=-\pgflinewidth, row sep=0.75mm, nodes in empty cells,
				row 1/.style={nodes={draw=none, fill=none, minimum size=4mm}},
				nodes in empty cells,
				row 2/.style={nodes={draw, fill=white, minimum size=6mm}}}]
				\matrix[array] (array) {
				1 & 2 & 3 & 4 & 5 & 6\\
				0&1&1&1&3&3\\};
				\begin{scope}[on background layer]
					\fill[green!20] (array-1-1.north west) rectangle (array-1-6.south east);
				\end{scope}
				\draw (array-1-1.west) node [left]{indices};
				\draw (array-2-1.west) node [left]{$S$};
			\end{tikzpicture}
		\end{center}
\end{enumerate}
\vspace*{18pt}
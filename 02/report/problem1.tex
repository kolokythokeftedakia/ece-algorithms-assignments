\begin{problem}
	\label{problem_1}
	Ένα δίκτυο υπολογιστών αναπαρίσταται από έναν γράφο $G = (V, E)$ όπου οι κόμβοι αναπαριστούν συσκευές και οι ακμές αναπαριστούν τις συνδέσεις μεταξύ των συσκευών. Κάθε ακμή $(u, v)$ έχει ένα βάρος το οποίο εκφράζει την πιθανότητα $p_{uv}$ ότι ένα πακέτο το οποίο στέλνεται από τη συσκευή $u$ θα φτάσει στην συσκευή $v$ χωρίς να χαθεί. Οι πιθανότητες είναι ανεξάρτητες. Ζητείται το μονοπάτι από την συσκευή $s$ προς τη συσκευή $t$ με τη μέγιστη πιθανότητα επιτυχούς αποστολής.
\end{problem}

\begin{proposition}
	\label{proposition_1}
	Given a longest-paths problem, create a copy of the given edge-weighted DAG that is identical to the original, except that all edge weights are negated. Then the shortest path in this copy is the longest path in the original.\cit{Sedgewick}
\end{proposition}

Λαμβάνοντας υπ' όψιν τη φύση του Προβλήματος \ref{problem_1} και την Πρόταση \ref{proposition_1} προκύπτουν τα ακόλουθα συμπεράσματα. Αρχικά, εφ' όσον ο γράφος $G$ αναπαριστά ένα πραγματικό δίκτυο συσκευών, δεν δύναται να υπάρχει ακμή $(u,v)\in E$, μεταξύ δύο διαφορετικών κορυφών $u,v\in V$, η οποία να έχει βάρος $w\(\(u,v\)\)=p_{uv}=1$, διότι αυτό θα σήμαινε πως υπάρχει μηδενική πιθανότητα αποτυχίας της αποστολής σε ένα πραγματικό σύστημα. Επιπλέον, η άρνηση της συνάρτησης βάρους των ακμών του $G$ υπολογίζεται ως εξής
\begin{equation}
	\label{eq:1_negated_w}
	\forall e\in E:\;\lnot w(e)=1.0-w(e)>0.
\end{equation}
καθότι η πιθανότητα είναι φραγμένη ποσότητα στο διάστημα $[0,1.0]$ και έχουμε θεωρήσει πως το σύστημα δεν είναι ιδανικό $\(\nexists e\in E:w(e)=1\)$.\par
Έστω, ο γράφος $\lnot G=(V^{\prime},E{^\prime})$ με βάρη, τέτοιος ώστε $V^{\prime}=V$, $E^{\prime}=E$ και τα βάρη των ακμών του να δίνονται από την $w^{\prime}:E^{\prime}\to[0,1]$, όπου $w^{\prime}=\lnot w$. Η σχέση \eqref{eq:1_negated_w} εξασφαλίζει πως στον $\lnot G$ δεν υπάρχουν ακμές με βάρος $0$. Αυτό επιτρέπει τη χρήση της λογικής του αλγορίθμου του Dijkstra\footnote{Αν υπήρχαν ακμές με βάρος $0$, δεδομένου πως το μήκος του μονοπατιού θα υπολογιστεί χρήσει του γινομένου $p_{e_i}\cdot p_{e_{j}}$, θα υπήρχε ακμή μέσω της οποίας διαρκώς θα μπορούμε να παίρνουμε μονοπάτι με μήκος $0$ με συνέπεια η διαδικασία να είναι ατέρμων. Επιπλέον, η σχέση \eqref{eq:1_negated_w} επιτρέπει την επέκταση της Πρότασης \ref{proposition_1} σε μη κατευθυνόμενους γράφους με κυκλώματα.} για την εύρεση του μονοπατιού ελάχιστου μήκους ή στην προκειμένη ελάχιστης πιθανότητας αποτυχίας στον $\lnot G$ και κατά συνέπεια, βάσει της Πρότασης \ref{proposition_1}, της διαδρομής μέγιστης πιθανότητας επιτυχούς αποστολής στον $G$.\par
Προτείνουμε τον Αλγόριθμο \ref{alg:2_1}. Δέχεται ως είσοδο έναν γράφο $G=(V,E)$ ο οποίος αναπαρίσταται με adjacency lists τα οποία επιτρέπουν την πρόσβαση στη γειτονιά μιας κορυφής σε γραμμικό χρόνο ως προς το μέγεθος της γειτονιάς \cit{algDesign}, μία κορυφή $s$ η οποία είναι το αρχικό άκρο του ζητούμενου μονοπατιού, μία κορυφή $t$ η οποία είναι το τερματικό άκρο του μονοπατιού και την συνάρτηση βαρών των ακμών $w:E\to[0,1]$.\par
Στις γραμμές $2-3$ υπολογίζεται το negated βάρος όλων των ακμών. Το σώμα του βρόχου εκτελείται $|E|$ φορές και κάθε εκτέλεσή του χρειάζεται σταθερό χρόνο $\bigO(1)$. Επομένως, ο βρόχος επανάληψης των γραμμών $2-3$ χρειάζεται $\bigO(|E|)$ χρόνο.\par
Έπειτα, στη γραμμή $4$ δημιουργείται ένα κενό Fibonacci heap $\mathcal{F}$ σε χρόνο $\Theta(1)$ \cit{handbook}. Στην γραμμή 5 δημιουργείται ένας κενός πίνακας $\mathcal{P}$, στον οποίο θα αποθηκευτεί το ζητούμενο μονοπάτι, σε σταθερό χρόνο $\bigO(1)$. Στις γραμμές $6$ και $7$ δημιουργούνται δύο κενοί πίνακες, $\pi$ και $\mathcal{L}$ αντιστοίχως, μεγέθους $|V|$. Άρα, ο χρόνος εκτέλεσης των γραμμών $5-7$ είναι $3\bigO(1)=\bigO(1)$.\par
Στις γραμμές $8$ και $9$ πραγματοποιούνται δύο καταχωρήσεις, μία στον πίνακα $\pi$ και μία στον πίνακα $\mathcal{L}$, οι οποίες απαιτούν σταθερό χρόνο. Στις γραμμές $10-12$ αρχικοποιούμε τις υπόλοιπες θέσεις των προαναφερθέντων πινάκων. Το σώμα του $\tw{for}$ εκτελείται $|V|-1$ φορές και κάθε φορά απαιτεί χρόνο $2\bigO(1)=\bigO(1)$. Άρα, συνολικά οι γραμμές $8$ έως και $12$ απαιτούν $\bigO(|V|)$ χρόνο.\par
Στις γραμμές $13-14$ εισάγουμε στο Fibonacci heap όλες τις κορυφές του γράφου $G$ χρησιμοποιώντας ως κλειδί την τιμή που τους αντιστοιχεί στον πίνακα $\mathcal{L}$. Η μέθοδος $\tw{insert(\cdot,\cdot)}$ για ένα Fibonacci heap απαιτεί $\Theta(1)$ \cit{handbook}. Επομένως, οι γραμμές $13-14$ χρειάζονται $\Theta(|V|)$ χρόνο.\par
Στην γραμμή $16$ η μέθοδος $\tw{extract\_min()}$ που αφορά το Fibonaccci heap έχει χρόνο εκτέλεσης $\bigO(\lg{|V|})$ \cit{handbook}. Η κατάχωρηση απαιτεί σταθερό χρόνο και επομένως ο συνολικός χρόνος κάθε εκτέλεσης της γραμμής $16$ είναι $\bigO(\lg{|V|})$. Το σώμα του $\tw{while}$ μπορεί να εκτελεσθεί έως και $|V|$ φορές.\par
\texttt{\vspace*{-11pt}
\begin{algorithm}[H]
	\caption{The algorithm is based on the implementation of Dijkstra's algorithm by T. H. Cormen and C. E. Leiserson and R. L. Rivest and C. Stein.\cit{handbook}}\label{alg:2_1}
	\begin{algorithmic}[1]
		\Procedure{MAXPP}{$G,s,t,w$}
			\ForAll{$e\in E$}
				\State{$w(e)\leftarrow 1.0-w(e)$}
			\EndFor
			\State{$\typeID{FibonacciHeap}\;\mathcal{F}\leftarrow\emptyset$}
			\State{$\typeID{array}\;\mathcal{P}\leftarrow\emptyset$}
			\State{$\typeID{array}\;\pi\leftarrow\emptyset\cdot|V|$}
			\State{$\typeID{array}\;\mathcal{L}\leftarrow\emptyset\cdot|V|$}
			\State{$\pi [s]\leftarrow \nil$}
			\State{$\mathcal{L}[s]\leftarrow 0$}
			\ForAll{$v\in V-\{s\}$}
				\State{$\pi[v]\leftarrow \nil$}
				\State{$\mathcal{L}[v]\leftarrow 1.0$}
			\EndFor
			\ForAll{$v\in V$}
				\State{$\mathcal{F}.\tw{insert}(v,\mathcal{L}[v])$}
			\EndFor
			\While{$\mathcal{F}\neq\emptyset$}
				\State{$u\leftarrow \mathcal{F}.\tw{extract\_min()}$}
				\State{$\mathcal{P}\leftarrow\mathcal{P}\cup\{u\}$}
				\ForAll{$v\in u.\tw{AdjLst}\;\AND\;u\neq s$}
					\If{$\mathcal{L}[v]>\mathcal{L}[u]\cdot w(u,v)$}
						\State{$\mathcal{L}[v]\leftarrow \mathcal{L}[u]\cdot w(u,v)$}
						\State{$\pi[v]\leftarrow u$}
						\State{$\mathcal{F}.\tw{decrease\_key}(v,\mathcal{L}[v])$}
					\EndIf
				\EndFor
				\If{$u==s$}
					\ForAll{$v\in u.\tw{AdjLst}$}
						\If{$\mathcal{L}[v]>w(u,v)$}
							\State{$\mathcal{L}[v]\leftarrow w(u,v)$}
							\State{$\pi[v]\leftarrow u$}
							\State{$\mathcal{F}.\tw{decrease\_key}(v,\mathcal{L}[v])$}
						\EndIf
					\EndFor
				\EndIf
			\EndWhile
			\State{\Return{$\mathcal{P},\pi$}}
		\EndProcedure
	\end{algorithmic}
\end{algorithm}}
Η γραμμή $17$ εκτελείται κάθε φορά σε σταθερό χρόνο. Σε σταθερό χρόνο εκτελείται και κάθε μία από τις γραμμές $19$, $20$, $21$ και $22$ \cit{handbook}. Το σώμα της $\tw{for}$ των γραμμών $18-22$ εκτελείται το πολύ $|E|$ φορές. Άρα ο χρόνος εκτέλεσης είναι $\bigO(|E|)$.\par
Η γραμμή $23$ εκτελείται $|V|$ φορές σε σταθερό χρόνο ενώ το σώμα της (γραμμές $24-28$) εκτελείται μία φορά μόνο. Το σώμα της $\tw{for}$ εκτελείται το πολύ $|E|$ φορές και κάθε μία από τις γραμμές $25-28$ χρειάζεται σταθερό χρόνο. Επομένως, οι γραμμές $23-28$ απαιτούν $\bigO(|E|)$ χρόνο.\par
Τέλος, η γραμμή $29$ έχει σταθερό χρόνο εκτέλεσης $\bigO(1)$. Βάσει όλων των παραπάνω, ο χρόνος εκτέλεσης του Αλγορίθμου \ref{alg:2_1} είναι $T\(|V|,|E|\)=\bigO(|E|)+\bigO(|V|)+\bigO(|V|\lg{|V|}+|E|)$, δηλαδή
\begin{equation}
	\label{eq:T(n,m)_prblm1}
	T\(|V|,|E|\)=\bigO(|V|\lg{\(|V|\)}+|E|).
\end{equation}

Η σχέση \eqref{eq:T(n,m)_prblm1} είναι σύμφωνη με τον χρόνο εκτέλεσης του αλγορίθμου του Dijkstra με Fibonaccci heap. \cit{dasgupta}\cit{algDesign}\cit{handbook}

Το μονοπάτι ελάχιστης πιθανότητας αποτυχούς αποστολής στον $\lnot G$ το οποίο ταυτίζεται με το μονοπάτι μέγιστης πιθανότητας επιτυχημένης αποστολής είναι αποθηκευμένο με τη μορφή κορυφών στον πίνακα $\mathcal{P}=\left\langle s,\ldots,v_i,\ldots,t\right\rangle$. Στον πίνακα $\pi$ είναι αποθηκευμένος για κάθε κορυφή $v$ του $G$ ο γονέας της στο μονοπάτι $\mathcal{P}$ (εάν η $v$ είναι η $s$ ή $v\notin\mathcal{P}$, τότε $\pi[v]=\nil$).